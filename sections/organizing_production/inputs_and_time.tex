\subsection{Inputs and Time}
\label{sec: inputs and time}

We have not considered possible material inputs needed for production so far.
The only input necessary was time. But people typically do not build things
completely from the ground up. A different person typically gathers materials
than the person who uses them in production. So we need to model inputs. If we
allow \(X_i\) to be outside of \(\real_{\ge 0}^\dims\), negative entries
represent the consumption of some good for production.

\subsubsection{Variable Cost Production with Inputs}

Let us now consider a specific example. 
\[
	h_{\cdot k} = (h_{1k},\dots, h_{\dims k})\]
is production method \(k\). Completing one cycle of production results in the
production of \(h_{jk}\) units of good \(j\). If \(h_{jk}\) is negative, this
was an input. Typically most entries are zero, some are negative and one is
positive. Let \(v = (v_k)_{k\le m}\) denote the number of production cycles for
each production method. Then for \(H=(h_{jk})_{j\le\dims,k\le m}\) our total
production is given by
\[
	x = Hv.
\]
Let \(l_i^{(k)}\) denote the labour requirements of person \(i\) to complete
one cycle of production method \(k\). Then their production capabilities are
given by
\[
	X_i = \low(\{ Hv : v\in \real_{\ge 0}^\dims, \langle v, l_i\rangle \le L_i\}).
\]
where \(\low(M):=\{x\in \real^{\dims}: \exists y\in M,\; x\precsim y\}\) denotes
the smallest set containing \(M\) which is a \ref{eq: lower layer}. If one does
not want to allow partial production cycles, one could also require
\(v\in\nat^\dims\). Although this would result in a difficult unbounded
Knapsack problem once we state the income maximization problem
\[
	\mu(p, X_i)	= \sup_{y\in X_i} \langle y, p\rangle.
\]
The greedy approach to the Knapsack problem
would convert this problem back into the continuous case, which we can solve
explicitly as follows:

\begin{lemma}[Income in Variable Cost Production with Inputs Case]
	For reasonable prices \(p\in \real_{\ge 0}^\dims\), we have
	\[
		\mu(p, X_i)
		= L_i \max_{j=1,\dots,d}
		\underbrace{\frac{(H^T p)_k}{l_i^{(k)}}}_{=:w_i^{(k)}}
		= L_i \underbrace{\|w_i\|_\infty}_{\text{wage}}.
	\]
	\((H^T p)_k\) is the net profit of production method \(k\) before labour
	costs. In other words, revenue from the sale of outputs minus cost of inputs.
\end{lemma}
\begin{proof}
	We can assume without loss of generality
	\[
		X_i = \{ Hv : v\in \real_{\ge 0}^\dims, \langle v, l_i\rangle \le L_i\},
	\]
	as the income from any \(y\precsim x\) with \(x\in X_i\) is lower or equal
	to the income from \(x\). Then we have
	\[\begin{aligned}
		\mu(p, X_i)
		&= \sup\{
			\langle Hv, p\rangle :
			v\in \real_{\ge 0}^\dims,\; \langle v, l_i\rangle \le L_i
		\}\\
		&= \sup\{
			\langle v, H^T p\rangle :
			v\in \real_{\ge 0}^\dims,\; \langle v, l_i\rangle \le L_i
		\}
	\end{aligned}\]
	From here the proof is the same as in Lemma~\ref{lem: income in the pure variable cost case}
	using \(H^T p\) in place of \(p\).
\end{proof}

Adding fixed costs \(f^{(k)}_i\) to enable production method \(k\) for person
\(i\) is a similarly straightforward generalization from Lemma~\ref{lem: income
in mixed case} resulting in average costs \(\bar{l}_i^{(k)}\).
