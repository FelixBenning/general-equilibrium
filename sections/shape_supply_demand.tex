\section{Shape of Supply \& Demand}
\label{sec: shape supply/demand}

First let us properly define supply and demand. The demand of individual \(i\)
is given by
\[
	D_i(p) := \{
		y : (L,y)
		\text{ is a solution to \eqref{eq: individual decision problem}}
	\}.
\]
Notice that it is a set, since there might be multiple solutions. The same is
true for the supply of individual \(i\)
\[
	S_i(p) := \argmax_{x\in X_i} \langle p, x\rangle 
	\overset{\text{Lem.~\ref{lem: subgradient of support functions}}}=
	\partial_p \mu(p, X_i).
\]
We can then define total demand and supply
\[
	D(p) := \sum_{i=1}^n D_i(p),
	\qquad
	S(p) := \sum_{i=1}^n S_i(p),
\]
where the sums are Minkowski sums.
And \(p^*\) is a market equilibrium, if
\[
	S(p^*)\cap D(p^*)\neq\emptyset.
\]
If \(S(p^*)\) and \(D(p^*)\) only contain a single element, one may write
\(S(p^*)=D(p^*)\).

\subsection{Shape of Supply}

We are interested in the average supply
\[
	\tfrac1n S(p) = \frac1n \sum_{i=1}^n S_i(p)
	= \frac1n\sum_{i=1}^n \partial_p\mu(p, X_i)
	= \partial_p \tfrac1n \mu(p, X)
	= \partial_p \mu(p, \bar{X}_n),
\]
where we have used the linearity of support functions with regards to Minkowski
sums and linearity of the subgradient in the last two equations. We are now
going to delay taking the subgradient until later and consider the average
income
\[
	\mu(p,\bar{X}_n)
	= \frac1n \sum_{i=1}^n \mu(p, X_i)
	= \frac1n \sum_{i=1}^n L_i \|w_i\|_\infty
	\overset{\text{SLLN}}\to \E[L_1 \|w_1\|_\infty] \quad (n\to\infty).
\]
Where it is not unreasonable to assume that the work times and wages
\(L_i\|w_i\|_\infty\) are iid random variables as we draw from a population of
people.

Unfortunately we now have lost all dependence on \(p\) which makes it impossible
to take the derivative. To enable us to do this, we need some form of parametric
model. So we turn to our fixed \& variable cost mixed case. Here we assume that
the average time requirements \(\bar{l}_i= (\bar{l}_i^{(1)}, \dots,
\bar{l}_i^{(\dims)})\) are iid random vectors. But as it will make notation
later on easier, we are going to convert these average time requirements into
(similarly independent) units per time
\[
	Q = (Q_1,\dots, Q_\dims)
	= \Bigl(\frac1{\bar{l}_1^{(1)}}, \dots, \frac1{\bar{l}_1^{(\dims)}}\Bigr).
\]
Furthermore we are going to assume \(L_i=L_1\) is constant, which is not
completely unreasonable given \(40\) hour work weeks\footnote{
	We only really use this to move it out of the expectation so you could
	alternatively assume independence to factorize the expectation. While this
	is mathematically more general since constant random variables are trivially
	independent from all others, it does not offer a real generalization. Because
	we already argued that wages are increasing in \(L_i\). So if \(L_i\) was
	not constant it certainly is not independent from \(\|w_i\|_\infty\).
}.
Finally we use the well known formula for the expectation of positive random
variables\footnote{
	Let \(X\) be an almost surely positive random variable, then
	\[
		\E[X]	= \int_0^\infty x d\Pr_X(x)
		\overset{\text{Fub.}}= \int_0^\infty \int_0^\infty \ind_{t < x}d\Pr_X(x)dt
		= \int_0^\infty 1- \Pr(X \le t)dt.
	\]
} to get
\begin{align*}
	\E[L_1\|w_1\|_\infty]
	&= L_1 \E[\|w_1\|_\infty]
	= L_1 \int_0^\infty 1 - \Pr(\|w_1\|_\infty \le x)dx\\
	&= L_1 \int_0^\infty 1- \Pr\Bigl(\max_j \frac{p_j}{\bar{l}_1^{(j)}} \le x\Bigr)dx\\
	&= L_1 \int_0^\infty 1- 
	\underbrace{
		\Pr\Bigl(Q_1 \le \frac{x}{p_1}, \dots, Q_\dims \le \frac{x}{p_\dims}\Bigr)
	}_{=: F(\frac{x}{p_1},\dots,\frac{x}{p_\dims})}dx.
\end{align*}
We therefore get
\[
	\mu(p, \bar{X}_n)
	\to L_1\int_0^\infty 1- F\bigl(\tfrac{x}{p_1},\dots,\tfrac{x}{p_\dims}\bigr)dx
	=: h(p)
\]
Since \(h\) retains all the properties of a support function as a limit of
support functions, there exists some set \(X^\infty\) such that \(\mu(p,
X^\infty)=h(p)\)\fxnote{does the set limit hold we defined earlier?}. In this
sense we have \(\bar{X}_n\to X^\infty\). We now want to consider the limit
supply
\begin{align*}
	S_\infty(p)
	&:= \partial_p \mu(p, X^\infty)\\
	&= L_1 \int_0^\infty \nabla_p
	\Bigl(1-F\bigl(\tfrac{x}{p_1},\dots, \tfrac{x}{p_\dims}\bigr)\Bigr)dx\\
	&=L_1 \begin{pmatrix}
		\int_0^\infty 
		\frac{\partial}{\partial x_1}
		F\bigl(\tfrac{x}{p_1},\dots, \tfrac{x}{p_\dims}\bigr)\frac{x}{p_1^2}dx
		\\\vdots\\
		\int_0^\infty 
		\frac{\partial}{\partial x_\dims}
		F\bigl(\tfrac{x}{p_1},\dots, \tfrac{x}{p_\dims}\bigr)\frac{x}{p_\dims^2}dx
	\end{pmatrix}\\
	\overset{y=\frac{x}{p_j}}&=L_1 \begin{pmatrix}
		\int_0^\infty 
		\frac{\partial}{\partial x_1}
		F\bigl(\tfrac{p_1}{p_1}y,\dots, \tfrac{p_1}{p_\dims}y\bigr)ydy
		\\\vdots\\
		\int_0^\infty 
		\frac{\partial}{\partial x_\dims}
		F\bigl(\tfrac{p_\dims}{p_1}y,\dots, \tfrac{p_\dims}{p_\dims}y\bigr)ydy
	\end{pmatrix},
\end{align*}
where we have somewhat optimistically swapped differentiation with integration
and replaced the subgradient with an actual gradient assuming that \(F\) is
differentiable and the subgradient thus unique. Let us now assume \(F\) to be
absolutely continuous with density \(f\). Then keeping in mind that \(Q_k\) are
positive
\begin{align*}
	\frac{\partial}{\partial x_\dims}F(x_1,\dots, x_\dims)
	&= \int_0^{x_1} \dots \int_0^{x_{\dims-1}}
	f(y_1,\dots, y_{\dims-1},x_\dims) dy_1\dots dy_{\dims-1}\\
	&= \idotsint \prod_{j=1}^{\dims-1} \ind_{y_j\le x_j}
	f(y_1,\dots, y_{\dims-1},x_\dims) dy_1\dots dy_{\dims-1}.
\end{align*}
So the supply for product \(\dims\) converges to
\begin{align*}
	S_\infty^{(\dims)}(p) 
	&=L_1\int_0^\infty
	y_\dims
	\frac{\partial}{\partial x_\dims} F\bigl(
		\tfrac{p_\dims}{p_1}y_\dims,
		\dots,
		\tfrac{p_\dims}{p_{\dims-1}}y_\dims, y_\dims
	\bigr) dy_\dims\\
	&= L_1 \idotsint y_\dims\prod_{j=1}^{\dims-1}\ind_{y_j\le \frac{p_\dims}{p_j}y_\dims}
	f(y_1,\dots,y_\dims)dy_1\dots dy_\dims\\
	&= L_1 \E\Bigl[
		Q_\dims \prod_{j=1}^{\dims-1}\ind_{p_jQ_j\le p_\dims Q_\dims}
	\Bigr].
\end{align*}
Similarly we get for all other products
\[
	S_\infty^{(k)}(p)
	= L_1 \E\Bigl[
		Q_k \prod_{\substack{j\le\dims\\j\neq k}}\ind_{p_jQ_j\le p_k Q_k}
	\Bigr].
\]
It is quite obvious that the supply of unit \(k\), \(S^\infty_k(p)\), is
monotonously increasing in \(p_k\).  If we further assume \(\E[Q_k]<\infty\), we
get by dominated convergence
\[
	\lim_{p_k\to \infty}S_\infty^{(k)}(p) = L_1\E[Q_k]
	\qquad\text{and}\qquad
	\lim_{p_k\to 0}S_\infty^{(k)}(p) = 0.
\]
If you recall that \(Q_k\) denotes the units of \(k\) per time a random
person can produce, this makes perfect sense. If the price increases to infinity
all efforts would go towards producing good \(k\). So we multiply the labour
time by the average number of units of \(k\) a person produces to obtain the
average supply. Similarly nobody will produce good \(k\) if the price goes to
zero.

\subsection{Shape of Demand}

\subsubsection{One or None Model}

Let us consider discrete products of which people might want to buy a single
item or none.

Let the willingness to work of person \(i\) for product \(j\) be given by
\(Z_i^{(j)}\). Taking the simpler willingness to pay from the pure variable case
we get
\[
	\wtp_i^{(j)}
	= Z_i^{(j)}\|w_i\|_\infty
	=\max\{ \underbrace{Z_i^{(j)}\|w_i^{(-j)}\|_\infty}_{=:\text{wtb}_i^{(j)}},  Z_i^{(j)}w_i^{(j)}\},
\]
where we remove \(w_i^{(j)}\) from \(w_i\) to obtain
\[
	w_i^{(-j)} := (w_i^{(1)}, \dots, \widehat{w_i^{(j)}}, \dots, w_i^{(\dims)}),
\]
and defined the ``willingness to buy'' \(\text{wtb}_i^{(j)}\).
The demand of person \(i\) for product \(j\) is therefore given by
\[\begin{aligned}
	D_i^{(j)}(p)
	&= \ind_{\wtp_i^{(j)} > p_j}
	= \ind_{\wtb_i^{(j)} > p_j \text{ or } Z_i^{(j)}w_i^{(j)} > p_j}\\
	&= 1- \ind_{\wtb_i^{(j)} \le p_j}\ind_{Z_i^{(j)}w_i^{(j)} \le p_j}\\
	\overset{w_i^{(j)}= \frac{p_j}{l_i^{(j)}}}&=
	1- \ind_{\wtb_i^{(j)} \le p_j}
	\ind_{Z_i^{(j)}\le l_i^{(j)}}.
\end{aligned}\]
The average demand for product \(j\) is then given by
\begin{align*}
	\tfrac1n D^{(j)}(p)
	&= \frac1n\sum_{i=1}^n(
		1- \ind_{\wtb_i^{(j)}\le p_j}
		\ind_{Z_i^{(j)}\le l_i^{(j)}}
	)\\
	&\to\begin{aligned}[t]
		D_\infty^{(j)}(p)
		&:= 1- \Pr(
			\wtb_1^{(j)} \le p_j,\;
			Z_1^{(j)}\le l_1^{(j)}
		)\\
		&=\underbrace{1- \Pr(\wtb_1^{(j)} \le p_j)}_{
			\begin{aligned}[t]
				&=1-F_{\wtb_1^{(j)}}(p_j)\\
				&=\bar{F}_{\wtb_1^{(j)}}(p_j)
			\end{aligned}
		} +\underbrace{
			\Pr(\wtb_1^{(j)} \le p_j,\; Z_1^{(j)} > l_1^{(j)})
		}_{
			\text{too expensive, but willing to DIY}
		}.
	\end{aligned}
\end{align*}
Here \(F_{\wtb_1^{(j)}}\) denotes the cumulative distribution function of the
willingness to buy for product \(j\). The Glivenko-Cantelli theorem can be
adapted to this case to prove the convergence is uniform in \(p_j\), since it
only uses monotonicity.

If the probability that people are willing to do it themselves (DIY) is small,
which is likely for a large number of products, then due to
\[
	0
	\le \underbrace{\Pr(\wtb_1^{(j)} \le p_j,\; Z_1^{(j)} > l_1^{(j)})}_{
		0\; \xleftarrow{p_j\downarrow 0}
		\qquad \xrightarrow{p_j\uparrow\infty}\; \mathrlap{\Pr(Z_1^{(j)} > l_1^{(j)})}
	}
	\le \Pr(Z_1^{(j)} > l_1^{(j)}) = \Pr(\text{willing to DIY}),
\]
the average demand in \(p_j\) is approximately equal to
\(\bar{F}_{\wtb_1^{(j)}}\). But while \(\wtb_1^{(j)}\) does not include \(p_j\)
it does include the remaining entries of \(p\), so it is not trivial to put
these back together.

\subsubsection{Fixed Budget Model}

In this section we want to model infinitely divisible products. For this we will
assume that every person \(i\) has a fixed work time budget \(L_i^{(j)}\) for
every good \(j\) and uses it up completely no matter the price. The total
work time is therefore \(L_i = \sum_{j=1}^\dims L_i^{(j)}\).
The money budget is then given by \(B_i^{(j)} = L_i^{(j)}\|w_i\|_\infty\). The
amount of \(j\) consumed by \(i\) is then given by
\begin{align}
	\label{eq: fixed budget and consumption relation}
	D_i^{(j)}(p)
	&= \frac{B_i^{(j)}}{p_j} = \frac{\|w_i\|_\infty}{p_j}L_i^{(j)}\\
	\label{eq: monotonously decreasing in pj}
	&= \max\Bigl\{
		\frac{\|w_i^{(-j)}\|_\infty}{p_j},\;
		\frac1{l_i^{(j)}}
	\Bigr\}L_i^{(j)}\\
	\nonumber
	&= \frac{\|w_i^{(-j)}\|_\infty}{p_j}L_i^{(j)}
	\ind_{\|w_i^{(-j)}\|_\infty l_i^{(j)} \ge p_j}
	+ \frac{L_i^{(j)}}{l_i^{(j)}}
	\ind_{\|w_i^{(-j)}\|_\infty l_i^{(j)} < p_j}.
\end{align}
As a sum of monotonously decreasing function (cf.~\eqref{eq: monotonously
decreasing in pj}) the average demand is also monotonously decreasing in \(p_j\)
and given by
\begin{align*}
	\tfrac1n D^{(j)}(p)
	&= \frac1n \sum_{i=1}^n D_i^{(j)}(p)\\
	&\to\begin{aligned}[t]
		&D_\infty^{(j)}(p)\\
		&:= \frac{\E\Bigl[
			L_i^{(j)}\|w_i^{(-j)}\|_\infty 
			\ind_{\|w_i^{(-j)}\|_\infty l_i^{(j)} \ge p_j}
		\Bigr]}{p_j}
		+ \E\Bigl[
			\frac{L_i^{(j)}}{l_i^{(j)}}
			\ind_{\|w_i^{(-j)}\|_\infty l_i^{(j)} < p_j}
		\Bigr].
	\end{aligned}
\end{align*}


\begin{lemma}[Cobb-Douglas Utility]
	We consider the logarithmic version of the Cobb-Douglas utility\footnote{
		The two variants are interchangeable in the utility optimization problem
		as the maximization problem does not change under monotonous transforms.
		But the logarithmic form actually has decreasing marginal utility. This is
		more realistic than the standard form should we introduce expected utility
		as it causes risk aversion.
	}
	\[
		u(f,x) = \lambda_0 \ln(f) + \sum_{j=1}^\dims \lambda_j \ln(x^{(j)})
	\]
	where \(f\) denotes free-time and we normalize time to \(1=f+L_i\). We
	further assume	 \(\frac{d}{dL_i}\|w_i\|_\infty=0\) (e.g. the variable cost
	case). Then work time is independent of prices
	\begin{equation}\label{eq: work time under Cobb-Douglas}
		L_i = 1- \underbrace{\frac{\lambda_0}{\sum_{j=0}^\dims \lambda_j}}_{=f}
		= \frac{\sum_{j=1}^\dims \lambda_j}{\sum_{j=0}^\dims \lambda_j}.
	\end{equation}
	Even without this assumption on the derivative of wages we obtain
	the first equation
	\[
		L_i^{(j)}
		= L_i\frac{\lambda_j}{\sum_{k=1}^\dims\lambda_k}
		\overset{\eqref{eq: work time under Cobb-Douglas}}=
		\frac{\lambda_j}{\sum_{k=0}^\dims\lambda_k},
	\]
	resulting in  \(L_i = \sum_{j=1}^\dims L_i^{(j)}\). The second equation,
	which requires \(\frac{d}{dL_i}\|w_i\|_\infty=0\), implies the work time
	budget is fixed and independent of prices. If \(L_i\) is fixed and can
	not be decided over by the agent \(i\) (e.g. 40 hour work week), then
	the second equation does not hold, but we still have a fixed work time
	budget by the first equation.
\end{lemma}
\begin{proof}
	Let us first optimize over \(y\) for a fixed \(L_i\) in the individual
	optimization problem
	\[
		\max_{L_i, y} u(1-L_i, y)
		\quad \text{s.t.}\quad
		\langle p, y\rangle \le \mu(p, X_i(L_i)) = L_i \|w_i\|_\infty,
	\]
	and later optimize over \(L_i\). Taking the derivative of the Lagrange
	function
	\[
		\mathcal{L}	(y, \gamma)
		= u(1-L_i, y) - \gamma [\langle p, y\rangle - L_i\|w_i\|_\infty]
	\]
	results in
	\[
		0\overset!= \frac{d}{d y^{(j)}}u(1-L_i,y) - \gamma p_j
		= \frac{\lambda_j}{y^{(j)}} - \gamma p_j
	\]
	which leads to \(y^{(j)} = \frac{\lambda_j}{\gamma p_j}\). Taking the
	derivative with regards to the Lagrange multiplier \(\gamma\) leads to
	\[
		L_i\|w_i\|_\infty = \langle p, y\rangle
		= \sum_{j=1}^\dims p_j \frac{\lambda_j}{\gamma p_j}
		= \frac{\sum_{j=1}^\dims \lambda_j}\gamma
	\]
	and thus \(\gamma = \frac{\sum_{j=1}^\dims \lambda_j}{L_i\|w_i\|_\infty}\).
	This implies with \eqref{eq: fixed budget and consumption relation}
	\[
		y^{(j)}
		= \frac{\|w_i\|_\infty}{p_j}
		\underbrace{L_i\frac{\lambda_j}{\sum_{j=1}^\dims \lambda_j}}_{=L_i^{(j)}}.
	\]
	Now we uses this result to define the remaining utility function in \(L_i\) 
	to be optimized
	\[
		u(L_i) = \lambda_0 \ln(1-L_i) + \sum_{j=1}^\dims \lambda_j \left[
			\ln(\|w_i\|_\infty) + \ln(L_i) + \ln\Bigl(\frac{\lambda_j}{p_j\sum_{j=1}^\dims \lambda_j}\Bigr)
		\right]
	\]
	Finally we take the derivative with regards to \(L_i\) to get
	\[
		0\overset!= -\frac{\lambda_0}{1-L_i}
		+ \left[
			\frac{\frac{d}{dL_i}\|w_i\|_\infty}{\|w_i\|_\infty}
			+ \frac1{L_i}
		\right]\sum_{j=1}^\dims \lambda_j.
	\]
	Using \(\frac{d}{dL_i}\|w_i\|_\infty = 0\) 
	and multiplying the entire equation by \(L_i(1-L_i)\) we get
	\[
		0\overset!= -\lambda_0L_i + (1-L_i)\sum_{j=1}^\dims \lambda_j.
	\]
	This immediately leads to our final claim
	\[
		L_i = \frac{\sum_{j=1}^\dims \lambda_j}{\sum_{j=0}^\dims \lambda_j}.
		\qedhere
	\]
\end{proof}


