\section{Shape of Supply \& Demand}
\label{sec: shape supply/demand}

First let us properly define supply and demand. The demand of individual \(i\)
is given by
\[
	D_i(p) := \{
		y : (L,y)
		\text{ is a solution to \eqref{eq: individual decision problem}}
	\}.
\]
Notice that it is a set, since there might be multiple solutions. The same is
true for the supply of individual \(i\)
\[
	S_i(p) := \argmax_{x\in X_i} \langle p, x\rangle 
	\overset{\text{Lem.~\ref{lem: subgradient of support functions}}}=
	\partial_p \mu(p, X_i).
\]
We can then define total demand and supply
\[
	D(p) := \sum_{i=1}^n D_i(p),
	\qquad
	S(p) := \sum_{i=1}^n S_i(p),
\]
where the sums are Minkowski sums.
And \(p^*\) is a market equilibrium, if
\[
	S(p^*)\cap D(p^*)\neq\emptyset.
\]
If \(S(p^*)\) and \(D(p^*)\) only contain a single element, one may write
\(S(p^*)=D(p^*)\).

\subsection{Shape of Supply}

We are interested in the average supply
\[
	\tfrac1n S(p) = \frac1n \sum_{i=1}^n S_i(p)
	= \frac1n\sum_{i=1}^n \partial_p\mu(p, X_i)
	= \partial_p \tfrac1n \mu(p, X),
\]
where we have used the linearity of support functions with regards to Minkowski
sums and linearity of the subgradient in the last equation. We are now going to
delay taking the subgradient until later and consider the average income
\[
	\tfrac1n \mu(p,X)
	= \frac1n \sum_{i=1}^n \mu(p, X_i)
	= \frac1n \sum_{i=1}^n L_i \|w_i\|_\infty
	\overset{\text{SLLN}}\to \E[L_1 \|w_1\|_\infty] \quad (n\to\infty).
\]
Where it is not unreasonable to assume that the work times and wages
\(L_i\|w_i\|_\infty\) are iid random variables as we draw from a population of
people.

Unfortunately we now have lost all dependence on \(p\) which makes it impossible
to take the derivative. To enable us to do this, we need some form of parametric
model. So we turn to our fixed \& variable cost mixed case. Here we assume that
the average time requirements \(\bar{l}_i= (\bar{l}_i^{(1)}, \dots,
\bar{l}_i^{(\dims)})\) are iid random vectors. But as it will make notation
later on easier, we are going to convert these average time requirements into
(similarly independent) units per time
\[
	Q = (Q_1,\dots, Q_\dims)
	= \Bigl(\frac1{\bar{l}_1^{(1)}}, \dots, \frac1{\bar{l}_1^{(\dims)}}\Bigr).
\]
Furthermore we are going to assume \(L_i=L_1\) is constant, which is not
completely unreasonable given \(40\) hour work weeks\footnote{
	We only really use this to move it out of the expectation so you could
	alternatively assume independence to factorize the expectation. While this
	is mathematically more general since constant random variables are trivially
	independent from all others, it does not offer a real generalization. Because
	we already argued that wages are increasing in \(L_i\). So if \(L_i\) was
	not constant it certainly is not independent from \(\|w_i\|_\infty\).
}.
Finally we use the well known formula for the expectation of positive random
variables\footnote{
	Let \(X\) be an almost surely positive random variable, then
	\[
		\E[X]	= \int_0^\infty x d\Pr_X(x)
		\overset{\text{Fub.}}= \int_0^\infty \int_0^\infty \ind_{t < x}d\Pr_X(x)dt
		= \int_0^\infty 1- \Pr(X \le t)dt.
	\]
} to get
\begin{align*}
	\E[L_1\|w_1\|_\infty]
	&= L_1 \E[\|w_1\|_\infty]
	= L_1 \int_0^\infty 1 - \Pr(\|w_1\|_\infty \le x)dx\\
	&= L_1 \int_0^\infty 1- \Pr\Bigl(\max_j \frac{p_j}{\bar{l}_1^{(j)}} \le x\Bigr)dx\\
	&= L_1 \int_0^\infty 1- 
	\underbrace{
		\Pr\Bigl(Q_1 \le \frac{x}{p_1}, \dots, Q_\dims \le \frac{x}{p_\dims}\Bigr)
	}_{=: F(\frac{x}{p_1},\dots,\frac{x}{p_\dims})}dx.
\end{align*}
Optimistically exchanging limits and differentiation and replacing our
subgradient by a gradient, we get
\begin{align*}
	\tfrac1n S(p) &= \nabla_p \tfrac1n\mu(p, X)\\
	&\to L_1 \int_0^\infty \nabla_p
	\Bigl(1-F\bigl(\tfrac{x}{p_1},\dots, \tfrac{x}{p_\dims}\bigr)\Bigr)dx\\
	&=L_1 \begin{pmatrix}
		\int_0^\infty 
		\frac{\partial}{\partial x_1}
		F\bigl(\tfrac{x}{p_1},\dots, \tfrac{x}{p_\dims}\bigr)\frac{x}{p_1^2}dx
		\\\vdots\\
		\int_0^\infty 
		\frac{\partial}{\partial x_\dims}
		F\bigl(\tfrac{x}{p_1},\dots, \tfrac{x}{p_\dims}\bigr)\frac{x}{p_\dims^2}dx
	\end{pmatrix}\\
	\overset{y=\frac{x}{p_j}}&=L_1 \begin{pmatrix}
		\int_0^\infty 
		\frac{\partial}{\partial x_1}
		F\bigl(\tfrac{p_1}{p_1}y,\dots, \tfrac{p_1}{p_\dims}y\bigr)ydy
		\\\vdots\\
		\int_0^\infty 
		\frac{\partial}{\partial x_\dims}
		F\bigl(\tfrac{p_\dims}{p_1}y,\dots, \tfrac{p_\dims}{p_\dims}y\bigr)ydy
	\end{pmatrix}.
\end{align*}
Let us now assume \(F\) to be absolutely continuous with density \(f\). Then
\begin{align*}
	\frac{\partial}{\partial x_\dims}F(x_1,\dots, x_\dims)
	&= \int_0^{x_1} \dots \int_0^{x_{\dims-1}}
	f(y_1,\dots, y_{\dims-1},x_\dims) dy_1\dots dy_{\dims-1}\\
	&= \idotsint \prod_{j=1}^{\dims-1} \ind_{y_j\le x_j}
	f(y_1,\dots, y_{\dims-1},x_\dims) dy_1\dots dy_{\dims-1}.
\end{align*}
So the supply for product \(\dims\) converges to
\begin{align*}
	&L_1\int_0^\infty
	\frac{\partial}{\partial x_\dims}
	F\bigl(\tfrac{p_\dims}{p_1}y_\dims,\dots,\tfrac{p_\dims}{p_{\dims-1}}y_\dims, y_\dims\bigr)y_\dims dy_\dims\\
	&= L_1 \idotsint \prod_{j=1}^{\dims-1}y_\dims\ind_{y_j\le \frac{p_\dims}{p_j}y_\dims}
	f(y_1,\dots,y_\dims)dy_1\dots dy_\dims
\end{align*}