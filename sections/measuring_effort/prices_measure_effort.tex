\subsection{Prices as a Measure of Effort}

In the hopes that we justified ownership, trade and prices sufficiently, let us
consider this in more detail. We have suggested using \(l^{(j)}_i\) as prices
(i.e. the effort required for production).  This implies that one good \(j\) has
different prices, depending on the producer \(i\). Since we assume that there
are no quality differences, people will obviously buy the cheaper products
first. Let \(p_j\) be the largest price for which good \(j\) is ever sold.
Potential producers \(i\) with \(l^{(j)}_i > p_j\) refrain from producing good
\(j\) and focus their attention on other products.

The producers with \(l^{(j)}_i < p_j\) now see, that they could have sold their
goods for more. And since the cost of production \(l^{(j)}_i\) is private
information, they will likely claim that this cost went up a little next time.
I.e. in the long run, only one prices \(p_j\) per good \(j\) is stable.
\begin{quotation}
\noindent Price \(p_j\) roughly measures the effort required to produce \(j\) by
the least efficient producer of \(j\).
\end{quotation}

The question remains: What is the right price? Should we average the
\(l^{(j)}_i\) somehow?

Since we are implicitly selecting the cutoff of who is going to produce \(j\) by
setting \(p_j\) and we want to ensure supply to be equal to demand \(S=D\), it
turns out we have very little choice when it comes to prices anyway. Let us see
what happens for a fixed price vector. We want \(S=D\). This obviously implies
\begin{equation}
	\label{eq: supply gdp = demand gdp}
	0 =\langle S-D,p\rangle =  \sum_{i=1}^n \langle S_i - D_i, p\rangle
\end{equation}
A market based system on the other hand ensures
\[
	0 = \langle S_i - D_i, p\rangle
	= \underbrace{\langle S_i, p\rangle}_{\text{income}}
	- \underbrace{\langle D_i,p\rangle}_{\text{expenses}}
\]
which is more than sufficient for the sums in Equation~\eqref{eq: supply gdp =
demand gdp} to be equal, but not sufficient for \(S=D\). In fact, it only
ensures that the price is orthogonal to the demand/supply mismatch \(S-D\).
If there are \(d\) products, that leaves a \(d-1\) hyperplane.

But as we will see, there are some price vectors, which do ensure \(S=D\).
We call these economic equilibria.
