\subsection{Prices in Societies of Clones}

Let us consider the mixed cost case and assume that everyone has the exact same
production capabilities (i.e. \(\bar{l}^{(j)} = \bar{l}_i^{(j)}\) for all
\(i\)). Then everyone's potential wages are identical
\[
	w_i^{(j)} = \frac{p_j}{\bar{l}^{(j)}} =: w^{(j)}.
\]
People will therefore only produce good \(j\) if \(w^{(j)} = \| w\|_\infty\).
If a society wants to consume a little of every good, every good needs to be
produced. But this implies all \(w^{(j)}\) have to be identical to
\(\|w\|_\infty\). This implies
\[
	\|w\|_\infty = \frac{p_j}{\bar{l}^{(j)}},
\]
and thus \(p_j = \bar{l}^{(j)}\|w\|_\infty\). Up to the factor \(\|w\|_\infty\)
of wages, we therefore fully determined prices. And as we argued in
Section~\ref{subsec: money supply and price stability}, we lose this degree of
freedom to the money supply. In fact by
\[
	\langle p, S\rangle = MV = \text{GDP}
\]
we can determine wages to be
\[
	\|w\|_\infty = \frac{\text{GDP}}{\langle \bar{l}, S\rangle}
	= \frac{\text{sum of money transactions}}{\text{time required to produce }S}.
\]
Where the transactions might be denoted in \$, the time might be denoted in
hours \(h\) and the wage is therefore denoted in \(\$/h\). Notice how
\(60 \text{min}/h\) has almost the same format. This is no coincidence. Since
everyone takes the same amount of time \(\bar{l}^{(j)}\) to produce good \(j\)
the \(\$\) value is literally a unit of time.

Once you break the assumption that all \(\bar{l}_i^{(j)}\) are identical, this
ceases to be the case. An easy generalization is when \(\bar{l}_i^{(j)}\) are
simply multiples of one another. In this case, the wage options are also
multiples. So we still need to require that they are all equal otherwise some 
good is not produced. This finally results in
\[
	p_j = \bar{l}_i^{(j)}\|w_i\|_\infty.
\]
In this case \(\|w_i\|_\infty\) can still be used as a subjective unit of time
(different for every person \(i\)).

The takeaway is: if it is obvious how the effort behind good \(j\) should be
valued, then the only price equilibrium is in fact this obvious valuation.
Prices should therefore be viewed as measures of effort.

But typically people are not just better in everything. Which means that the
measure of value \(p\) will finally need to depart from the measure of time
\(\bar{l}_i\).
