\subsection{Money Supply and Price Stability}
\label{subsec: money supply and price stability}

Since everyone simply ensures that their income is equal to their expenses, i.e.
that \(S_i - D_i\) is orthogonal to \(p\)
\[
	0 = \langle S_i - D_i, p\rangle,
\]
scaling of \(p\) has no effect on anyone's decision. So if \(p^*\) is an
economic equilibrium, so is \(cp^*\) for every \(c\in \real_{>0}\). So it
appears we have a surplus of one degree of freedom for \(p\). As it will turn
out, we lose that degree of freedom to the money supply.
The gross domestic product (GDP) is defined to be the sum of all transactions.
One way to calculate it, is to sum up all sales made by each individual
\[
	\text{GDP} = \sum_{i=1}^n \langle p, S_i\rangle = \langle p, S\rangle.
\]
If our economy runs on money (instead of some tally) on the other hand, there
is typically a fixed supply \(M\). If people receive payment at the start of
the month (wages) and spend it during the month, then the entire money supply
will be used once per month. This is of course a bit too simplistic, but the
point here is, that the money velocity \(V\) (i.e. the amount of times the same
coin changes hands per month or other time unit) is typically externally fixed.
But if we want to sum all transactions, and we know the size of the money supply
and also its cycle velocity, then we already know the total size of all
transactions:
\[
	\text{GDP} = MV.
\]
So it is necessarily true, that 
\begin{equation}
	\label{eq: nominal GDP equation}
	\langle p, S\rangle = MV.	
\end{equation}
If the society now manages to produce \(rS\) more of everything, resulting
in \(\tilde{S} = (1+r) S = cS\) then prices necessarily need to adjust to
\(\tilde{p} = \tfrac1c p\) to ensure that \eqref{eq: nominal GDP equation}
remains true, i.e.
\[
	\langle \tilde{p}, \tilde{S}\rangle = \langle \tfrac1c p, cS\rangle = MV.
\]
So if an economy grows with fixed money supply and velocity prices have to
shrink at the same rate, resulting in deflation. If an economy contracts on the
other hand, inflation would be the result.

Inflation and Deflation causes funky effects if one is able to carry over
money from one period to another. Because under inflation, this devalues earlier
work causing people to consume as quickly as possible. Deflation (falling
prices) on the other hand causes people to delay consumption. If consumption is
delayed long enough, then production \(S\) will also shrink again. This reduces
deflation, but also causes the economy to shrink senselessly. While inflation
does not have such an effect on actual production, too much of it is still
undesirable. For this reason central banks usually target \(0-2\%\) of inflation
to steer clear of deflation.

Due to advancements in technology and population growth, growth of \(S\) is much
more common than shrinkage.  So to avoid deflation it is vital to increase
the money supply \(M\) in pace with \(S\) unless the money velocity changes for
some reason.

Cryptocurrencies typically do not allow quantity adjustments by design and are
therefore rubbish as a currency.

The gold standard (and similar) use the remaining degree of freedom to fix the
price of a single good (gold). This comes at the expense of price stability
of all other prices. Assuming the supply of gold remains fixed while the rest
of the economy grows, this too causes deflation. Discoveries of large quantities
of gold increase the money supply on the other hand, causing an inflationary
shock.

No matter what system an economy uses, the takeaway is, that the superfluous
degree of freedom in \(p\) is removed by the money supply. And since anyone can
create money (it is just debt after all), determining and steering this supply
is a science in itself making the work of central banks quite difficult.
