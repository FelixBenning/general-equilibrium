\subsection{Shape of Supply}

We are interested in the average supply
\[
	\tfrac1n S(p) = \frac1n \sum_{i=1}^n S_i(p)
	= \frac1n\sum_{i=1}^n \partial_p\mu(p, X_i)
	= \partial_p \tfrac1n \mu(p, X)
	= \partial_p \mu(p, \bar{X}_n),
\]
where we have used the linearity of support functions with regards to Minkowski
sums and linearity of the subgradient in the last two equations. We are now
going to delay taking the subgradient until later and consider the average
income
\[
	\mu(p,\bar{X}_n)
	= \frac1n \sum_{i=1}^n \mu(p, X_i)
	= \frac1n \sum_{i=1}^n L_i \|w_i\|_\infty
	\overset{\text{SLLN}}\to \E[L_1 \|w_1\|_\infty] \quad (n\to\infty).
\]
Where it is not unreasonable to assume that the work times and wages
\(L_i\|w_i\|_\infty\) are iid random variables as we draw from a population of
people.

Unfortunately we now have lost all dependence on \(p\) which makes it impossible
to take the derivative. To enable us to do this, we need some form of parametric
model. So we turn to our fixed \& variable cost mixed case. Here we assume that
the average time requirements \(\bar{l}_i= (\bar{l}_i^{(1)}, \dots,
\bar{l}_i^{(\dims)})\) are iid random vectors. But as it will make notation
later on easier, we are going to convert these average time requirements into
(similarly independent) units per time
\[
	Q = (Q_1,\dots, Q_\dims)
	= \Bigl(\frac1{\bar{l}_1^{(1)}}, \dots, \frac1{\bar{l}_1^{(\dims)}}\Bigr).
\]
Furthermore we are going to assume \(L_i=L_1\) is constant, which is not
completely unreasonable given \(40\) hour work weeks\footnote{
	We only really use this to move it out of the expectation so you could
	alternatively assume independence to factorize the expectation. While this
	is mathematically more general since constant random variables are trivially
	independent from all others, it does not offer a real generalization. Because
	we already argued that wages are increasing in \(L_i\). So if \(L_i\) was
	not constant it certainly is not independent from \(\|w_i\|_\infty\).
}.
Finally we use the well known formula for the expectation of positive random
variables\footnote{
	Let \(X\) be an almost surely positive random variable, then
	\[
		\E[X]	= \int_0^\infty x d\Pr_X(x)
		\overset{\text{Fub.}}= \int_0^\infty \int_0^\infty \ind_{t < x}d\Pr_X(x)dt
		= \int_0^\infty 1- \Pr(X \le t)dt.
	\]
} to get
\begin{align*}
	\E[L_1\|w_1\|_\infty]
	&= L_1 \E[\|w_1\|_\infty]
	= L_1 \int_0^\infty 1 - \Pr(\|w_1\|_\infty \le x)dx\\
	&= L_1 \int_0^\infty 1- \Pr\Bigl(\max_j \frac{p_j}{\bar{l}_1^{(j)}} \le x\Bigr)dx\\
	&= L_1 \int_0^\infty 1- 
	\underbrace{
		\Pr\Bigl(Q_1 \le \frac{x}{p_1}, \dots, Q_\dims \le \frac{x}{p_\dims}\Bigr)
	}_{=: F(\frac{x}{p_1},\dots,\frac{x}{p_\dims})}dx.
\end{align*}
We therefore get
\[
	\mu(p, \bar{X}_n)
	\to L_1\int_0^\infty 1- F\bigl(\tfrac{x}{p_1},\dots,\tfrac{x}{p_\dims}\bigr)dx
	=: h(p)
\]
Since \(h\) retains all the properties of a support function as a limit of
support functions, there exists some set \(X^\infty\) such that \(\mu(p,
X^\infty)=h(p)\)\fxnote{does the set limit hold we defined earlier?}. In this
sense we have \(\bar{X}_n\to X^\infty\). We now want to consider the limit
supply
\begin{align*}
	S_\infty(p)
	&:= \partial_p \mu(p, X^\infty)\\
	&= L_1 \int_0^\infty \nabla_p
	\Bigl(1-F\bigl(\tfrac{x}{p_1},\dots, \tfrac{x}{p_\dims}\bigr)\Bigr)dx\\
	&=L_1 \begin{pmatrix}
		\int_0^\infty 
		\frac{\partial}{\partial x_1}
		F\bigl(\tfrac{x}{p_1},\dots, \tfrac{x}{p_\dims}\bigr)\frac{x}{p_1^2}dx
		\\\vdots\\
		\int_0^\infty 
		\frac{\partial}{\partial x_\dims}
		F\bigl(\tfrac{x}{p_1},\dots, \tfrac{x}{p_\dims}\bigr)\frac{x}{p_\dims^2}dx
	\end{pmatrix}\\
	\overset{y=\frac{x}{p_j}}&=L_1 \begin{pmatrix}
		\int_0^\infty 
		\frac{\partial}{\partial x_1}
		F\bigl(\tfrac{p_1}{p_1}y,\dots, \tfrac{p_1}{p_\dims}y\bigr)ydy
		\\\vdots\\
		\int_0^\infty 
		\frac{\partial}{\partial x_\dims}
		F\bigl(\tfrac{p_\dims}{p_1}y,\dots, \tfrac{p_\dims}{p_\dims}y\bigr)ydy
	\end{pmatrix},
\end{align*}
where we have somewhat optimistically swapped differentiation with integration
and replaced the subgradient with an actual gradient assuming that \(F\) is
differentiable and the subgradient thus unique. Let us now assume \(F\) to be
absolutely continuous with density \(f\). Then keeping in mind that \(Q_k\) are
positive
\begin{align*}
	\frac{\partial}{\partial x_\dims}F(x_1,\dots, x_\dims)
	&= \int_0^{x_1} \dots \int_0^{x_{\dims-1}}
	f(y_1,\dots, y_{\dims-1},x_\dims) dy_1\dots dy_{\dims-1}\\
	&= \idotsint \prod_{j=1}^{\dims-1} \ind_{y_j\le x_j}
	f(y_1,\dots, y_{\dims-1},x_\dims) dy_1\dots dy_{\dims-1}.
\end{align*}
So the supply for product \(\dims\) converges to
\begin{align*}
	S_\infty^{(\dims)}(p) 
	&=L_1\int_0^\infty
	y_\dims
	\frac{\partial}{\partial x_\dims} F\bigl(
		\tfrac{p_\dims}{p_1}y_\dims,
		\dots,
		\tfrac{p_\dims}{p_{\dims-1}}y_\dims, y_\dims
	\bigr) dy_\dims\\
	&= L_1 \idotsint y_\dims\prod_{j=1}^{\dims-1}\ind_{y_j\le \frac{p_\dims}{p_j}y_\dims}
	f(y_1,\dots,y_\dims)dy_1\dots dy_\dims\\
	&= L_1 \E\Bigl[
		Q_\dims \prod_{j=1}^{\dims-1}\ind_{p_jQ_j\le p_\dims Q_\dims}
	\Bigr].
\end{align*}
Similarly we get for all other products
\[
	S_\infty^{(k)}(p)
	= L_1 \E\Bigl[
		Q_k \prod_{\substack{j\le\dims\\j\neq k}}\ind_{p_jQ_j\le p_k Q_k}
	\Bigr].
\]
It is quite obvious that the supply of unit \(k\), \(S^\infty_k(p)\), is
monotonously increasing in \(p_k\).  If we further assume \(\E[Q_k]<\infty\), we
get by dominated convergence
\[
	\lim_{p_k\to \infty}S_\infty^{(k)}(p) = L_1\E[Q_k]
	\qquad\text{and}\qquad
	\lim_{p_k\to 0}S_\infty^{(k)}(p) = 0.
\]
If you recall that \(Q_k\) denotes the units of \(k\) per time a random
person can produce, this makes perfect sense. If the price increases to infinity
all efforts would go towards producing good \(k\). So we multiply the labour
time by the average number of units of \(k\) a person produces to obtain the
average supply. Similarly nobody will produce good \(k\) if the price goes to
zero.
