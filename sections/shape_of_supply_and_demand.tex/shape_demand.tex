\subsection{Shape of Demand}

\subsubsection{One or None Model}

Let us consider discrete products of which people might want to buy a single
item or none.

Let the willingness to work of person \(i\) for product \(j\) be given by
\(Z_i^{(j)}\). Taking the simpler willingness to pay from the pure variable case
we get
\[
	\wtp_i^{(j)}
	= Z_i^{(j)}\|w_i\|_\infty
	=\max\{ \underbrace{Z_i^{(j)}\|w_i^{(-j)}\|_\infty}_{=:\text{wtb}_i^{(j)}},  Z_i^{(j)}w_i^{(j)}\},
\]
where we remove \(w_i^{(j)}\) from \(w_i\) to obtain
\[
	w_i^{(-j)} := (w_i^{(1)}, \dots, \widehat{w_i^{(j)}}, \dots, w_i^{(\dims)}),
\]
and defined the ``willingness to buy'' \(\text{wtb}_i^{(j)}\).
The demand of person \(i\) for product \(j\) is therefore given by
\[\begin{aligned}
	D_i^{(j)}(p)
	&= \ind_{\wtp_i^{(j)} > p_j}
	= \ind_{\wtb_i^{(j)} > p_j \text{ or } Z_i^{(j)}w_i^{(j)} > p_j}\\
	&= 1- \ind_{\wtb_i^{(j)} \le p_j}\ind_{Z_i^{(j)}w_i^{(j)} \le p_j}\\
	\overset{w_i^{(j)}= \frac{p_j}{l_i^{(j)}}}&=
	1- \ind_{\wtb_i^{(j)} \le p_j}
	\ind_{Z_i^{(j)}\le l_i^{(j)}}.
\end{aligned}\]
The average demand for product \(j\) is then given by
\begin{align*}
	\tfrac1n D^{(j)}(p)
	&= \frac1n\sum_{i=1}^n(
		1- \ind_{\wtb_i^{(j)}\le p_j}
		\ind_{Z_i^{(j)}\le l_i^{(j)}}
	)\\
	&\to\begin{aligned}[t]
		D_\infty^{(j)}(p)
		&:= 1- \Pr(
			\wtb_1^{(j)} \le p_j,\;
			Z_1^{(j)}\le l_1^{(j)}
		)\\
		&=\underbrace{1- \Pr(\wtb_1^{(j)} \le p_j)}_{
			\begin{aligned}[t]
				&=1-F_{\wtb_1^{(j)}}(p_j)\\
				&=\bar{F}_{\wtb_1^{(j)}}(p_j)
			\end{aligned}
		} +\underbrace{
			\Pr(\wtb_1^{(j)} \le p_j,\; Z_1^{(j)} > l_1^{(j)})
		}_{
			\text{too expensive, but willing to DIY}
		}.
	\end{aligned}
\end{align*}
Here \(F_{\wtb_1^{(j)}}\) denotes the cumulative distribution function of the
willingness to buy for product \(j\). The Glivenko-Cantelli theorem can be
adapted to this case to prove the convergence is uniform in \(p_j\), since it
only uses monotonicity.

If the probability that people are willing to do it themselves (DIY) is small,
which is likely for a large number of products, then due to
\[
	0
	\le \underbrace{\Pr(\wtb_1^{(j)} \le p_j,\; Z_1^{(j)} > l_1^{(j)})}_{
		0\; \xleftarrow{p_j\downarrow 0}
		\qquad \xrightarrow{p_j\uparrow\infty}\; \mathrlap{\Pr(Z_1^{(j)} > l_1^{(j)})}
	}
	\le \Pr(Z_1^{(j)} > l_1^{(j)}) = \Pr(\text{willing to DIY}),
\]
the average demand in \(p_j\) is approximately equal to
\(\bar{F}_{\wtb_1^{(j)}}\). But while \(\wtb_1^{(j)}\) does not include \(p_j\)
it does include the remaining entries of \(p\), so it is not trivial to put
these back together.

\subsubsection{Fixed Budget Model}

In this section we want to model infinitely divisible products. For this we will
assume that every person \(i\) has a fixed work time budget \(L_i^{(j)}\) for
every good \(j\) and uses it up completely no matter the price. The total
work time is therefore \(L_i = \sum_{j=1}^\dims L_i^{(j)}\).
The money budget is then given by \(B_i^{(j)} = L_i^{(j)}\|w_i\|_\infty\). The
amount of \(j\) consumed by \(i\) is then given by
\begin{align}
	\label{eq: fixed budget and consumption relation}
	D_i^{(j)}(p)
	&= \frac{B_i^{(j)}}{p_j} = \frac{\|w_i\|_\infty}{p_j}L_i^{(j)}\\
	\label{eq: monotonously decreasing in pj}
	&= \max\Bigl\{
		\frac{\|w_i^{(-j)}\|_\infty}{p_j},\;
		\frac1{l_i^{(j)}}
	\Bigr\}L_i^{(j)}\\
	\nonumber
	&= \frac{\|w_i^{(-j)}\|_\infty}{p_j}L_i^{(j)}
	\ind_{\|w_i^{(-j)}\|_\infty l_i^{(j)} \ge p_j}
	+ \frac{L_i^{(j)}}{l_i^{(j)}}
	\ind_{\|w_i^{(-j)}\|_\infty l_i^{(j)} < p_j}.
\end{align}
As a sum of monotonously decreasing function (cf.~\eqref{eq: monotonously
decreasing in pj}) the average demand is also monotonously decreasing in \(p_j\)
and given by
\begin{align*}
	\tfrac1n D^{(j)}(p)
	&= \frac1n \sum_{i=1}^n D_i^{(j)}(p)\\
	&\to\begin{aligned}[t]
		&D_\infty^{(j)}(p)\\
		&:= \underbrace{\frac{\E\Bigl[
			L_i^{(j)}\|w_i^{(-j)}\|_\infty 
			\ind_{\|w_i^{(-j)}\|_\infty \ge \frac{p_j}{l_i^{(j)}}}
		\Bigr]}{p_j}}_{
			\text{buyers}
		}
		+ \underbrace{\E\Bigl[
			\frac{L_i^{(j)}}{l_i^{(j)}}
			\ind_{\|w_i^{(-j)}\|_\infty  < \frac{p_j}{l_i^{(j)}}}
		\Bigr]}_{\text{DIY'ers}}.
	\end{aligned}
\end{align*}


\begin{lemma}[Cobb-Douglas Utility]
	We consider the logarithmic version of the Cobb-Douglas utility\footnote{
		The two variants are interchangeable in the utility optimization problem
		as the maximization problem does not change under monotonous transforms.
		But the logarithmic form actually has decreasing marginal utility. This is
		more realistic than the standard form should we introduce expected utility
		as it causes risk aversion.
	}
	\[
		u(f,x) = \lambda_0 \ln(f) + \sum_{j=1}^\dims \lambda_j \ln(x^{(j)})
	\]
	where \(f\) denotes free-time and we normalize time to \(1=f+L_i\). We
	further assume	 \(\frac{d}{dL_i}\|w_i\|_\infty=0\) (e.g. the variable cost
	case). Then work time is independent of prices
	\begin{equation}\label{eq: work time under Cobb-Douglas}
		L_i = 1- \underbrace{\frac{\lambda_0}{\sum_{j=0}^\dims \lambda_j}}_{=f}
		= \frac{\sum_{j=1}^\dims \lambda_j}{\sum_{j=0}^\dims \lambda_j}.
	\end{equation}
	Even without this assumption on the derivative of wages we obtain
	the first equation
	\[
		L_i^{(j)}
		= L_i\frac{\lambda_j}{\sum_{k=1}^\dims\lambda_k}
		\overset{\eqref{eq: work time under Cobb-Douglas}}=
		\frac{\lambda_j}{\sum_{k=0}^\dims\lambda_k},
	\]
	resulting in  \(L_i = \sum_{j=1}^\dims L_i^{(j)}\). The second equation,
	which requires \(\frac{d}{dL_i}\|w_i\|_\infty=0\), implies the work time
	budget is fixed and independent of prices. If \(L_i\) is fixed and can
	not be decided over by the agent \(i\) (e.g. 40 hour work week), then
	the second equation does not hold, but we still have a fixed work time
	budget by the first equation.
\end{lemma}
\begin{proof}
	Let us first optimize over \(y\) for a fixed \(L_i\) in the individual
	optimization problem
	\[
		\max_{L_i, y} u(1-L_i, y)
		\quad \text{s.t.}\quad
		\langle p, y\rangle \le \mu(p, X_i(L_i)) = L_i \|w_i\|_\infty,
	\]
	and later optimize over \(L_i\). Taking the derivative of the Lagrange
	function
	\[
		\mathcal{L}	(y, \gamma)
		= u(1-L_i, y) - \gamma [\langle p, y\rangle - L_i\|w_i\|_\infty]
	\]
	results in
	\[
		0\overset!= \frac{d}{d y^{(j)}}u(1-L_i,y) - \gamma p_j
		= \frac{\lambda_j}{y^{(j)}} - \gamma p_j
	\]
	which leads to \(y^{(j)} = \frac{\lambda_j}{\gamma p_j}\). Taking the
	derivative with regards to the Lagrange multiplier \(\gamma\) leads to
	\[
		L_i\|w_i\|_\infty = \langle p, y\rangle
		= \sum_{j=1}^\dims p_j \frac{\lambda_j}{\gamma p_j}
		= \frac{\sum_{j=1}^\dims \lambda_j}\gamma
	\]
	and thus \(\gamma = \frac{\sum_{j=1}^\dims \lambda_j}{L_i\|w_i\|_\infty}\).
	This implies with \eqref{eq: fixed budget and consumption relation}
	\[
		y^{(j)}
		= \frac{\|w_i\|_\infty}{p_j}
		\underbrace{L_i\frac{\lambda_j}{\sum_{j=1}^\dims \lambda_j}}_{=L_i^{(j)}}.
	\]
	Now we uses this result to define the remaining utility function in \(L_i\) 
	to be optimized
	\[
		u(L_i) = \lambda_0 \ln(1-L_i) + \sum_{j=1}^\dims \lambda_j \left[
			\ln(\|w_i\|_\infty) + \ln(L_i) + \ln\Bigl(\frac{\lambda_j}{p_j\sum_{j=1}^\dims \lambda_j}\Bigr)
		\right]
	\]
	Finally we take the derivative with regards to \(L_i\) to get
	\[
		0\overset!= -\frac{\lambda_0}{1-L_i}
		+ \left[
			\frac{\frac{d}{dL_i}\|w_i\|_\infty}{\|w_i\|_\infty}
			+ \frac1{L_i}
		\right]\sum_{j=1}^\dims \lambda_j.
	\]
	Using \(\frac{d}{dL_i}\|w_i\|_\infty = 0\) 
	and multiplying the entire equation by \(L_i(1-L_i)\) we get
	\[
		0\overset!= -\lambda_0L_i + (1-L_i)\sum_{j=1}^\dims \lambda_j.
	\]
	This immediately leads to our final claim
	\[
		L_i = \frac{\sum_{j=1}^\dims \lambda_j}{\sum_{j=0}^\dims \lambda_j}.
		\qedhere
	\]
\end{proof}

\subsubsection{Knapsack Consumption Model}

In this section, we want to model the consumption of experiences, e.g. movies.
As consumption requires time, we have durations \(D=(D_j)_{j\le \dims}\) to
consume product \(j\). Let the available products be \(x^{(j)}\in \{0,1\}\) and
let \(U_i=(U_i^{(j)})_{j\le \dims}\) be the utility person \(i\) derives from
the consumption of product \(j\). Then the utility function is of the form
\[
	u(f, x) = \max\{
		\langle U, y\rangle : \underbrace{y\in \{0,1\}^\dims,\; y\precsim x}_{\mathclap{\text{sub-selection for consumption}}},\;
		\overbrace{\langle D, y\rangle \le f}^{\text{time constraint}}
	\}.
\]
So our individual decision
\[
	\max_{x, L_i}u(1-L_i, x) \quad\text{s.t.}\quad
	\langle p, x\rangle \le \mu(p, X_i(L_i))
\]
is equivalent to
\[
	\max_{L_i}\; \max\{
		\langle U, y\rangle: y\in \{0,1\}^\dims,\;
		\langle D, y \rangle \le 1-L_i,\;
		\langle p, y\rangle \le \mu(p, X_i(L_i))
	\}.
\]
We therefore have to solve a ``2d Knapsack Problem'' for every \(L_i\) and then
optimize over \(L_i\).
Since this problem is NP-hard, we can not really solve them. But at the same
time individual \(i\) will struggle to solve it as well. So if we use a
heuristic it might capture \(i\)'s decisions perfectly if they use the same
heuristic. One very common heuristic is the Greedy heuristic.
