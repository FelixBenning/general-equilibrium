\section{Aggregate Production Capabilities}

The total production capabilities \eqref{eq: tpc} of a society of \(n\) people, is given by the
Minkowski sum of individual production capabilities
\begin{equation}
	\label{eq: tpc}\tag{TPC}
	X = \{ x_1 + \dots + x_n : x_i \in X_i\} = \sum_{i=1}^n X_i
\end{equation}

If we split the spoils of production equally between everyone, then the set of
possible consumption vectors is given by the average production capability
\begin{equation}
	\label{eq: apc}\tag{APC}
	\bar{X}_n = \tfrac1n X = \{\tfrac{x}n : x\in X\}.
\end{equation}

In a society of identical clones, this is what we would expect to happen.

\begin{lemma}[Clone Production Capabilities]
	In a society of clones, we have \(X_1=\dots = X_n\). Then we observe:
	\begin{enumerate}
		\item In general we have \(X_1\subseteq \bar{X}_n\).
		\item If we assume that \(X_1\) is convex, then
		\[
			\bar{X}_n = X_1
		\]
		\item Using only the \ref{eq: lower layer} property of \(X_1\), we get
		\[
			\bar{X}_n \to \conv(X_1),
		\]
		where \(\conv(M)\) is the convex hull\footnote{
			The convex hull is defined as the set of all convex combinations
			\[
				\conv(M)= \Bigl\{\sum_{i=1}^m \lambda_i x_i : m\in\nat,\ x_i \in M,\ \lambda_i\in[0,1],\ \sum_{i=1}\lambda_i =1\Bigr\}
			\]
		} of the set \(M\) and we define
		convergence of sets as follows: \(M_n\) converges to \(M\), if

		\begin{enumerate}
			\item\label{set-conv: seq} For all \(x\in M\) exists a sequence
			\(x_n\in M_n\) with \(x_n\to x\).

			\item\label{set-conv: incl} For all \(x\) in the interior \(M^\circ\)
			of \(M\), exists \(n_0\in\nat\) such that \(x\in M_n\) for all \(n\ge
			n_0\).
		\end{enumerate}
		The \ref{eq: lower layer} property is only needed for \ref{set-conv:
		incl}.
	\end{enumerate}
\end{lemma}

\begin{proof}
\begin{enumerate}
	\item For \(X_1 \subseteq \bar{X}_n\), we take \(x\in X_1\) and
	observe that if we pick 
	\[
		x_1=\dots=x_n=x
	\]
	in the Minkowski sum we obtain \(n x \in X\). So we have
	\(x\in\bar{X}_n\).
	

	\item
	Equality in the convex case is almost trivial as well. Let \(x\in
	\bar{X}_n\), then there exist \(x_1,\dots,x_n \in X_1\) such that \[
		x = \frac1n \sum_{i=1}^n x_i = \sum_{i=1}^n \frac1n x_i
		\overset{\text{convex}}\in X_1.
	\]

	\item
	\begin{enumerate}
		\item Pick \(x\in \conv(X_1)\). Since it is in the convex hull,
		there exist \(\lambda_1,\dots,\lambda_m\) summing to unity, \(x_i\in X_1\)
		with
		\[
			x = \sum_{i=1}^m \lambda_i x_i.
		\]
		We set the number of clones producing \(x_i\) to the largest integer smaller
		\(n\lambda_i\)
		\[
			k_i^{(n)} := \lfloor n\lambda_i \rfloor.
		\]
		This is possible due to	
		\[
			\sum_{i=1}^m k_i^{(n)} \le n \sum_{i=1} \lambda_i = n.
		\]
		This results in
		\[
			x^{(n)}:= \frac1n\sum_{i=1}^m k_i x_i \in \bar{X}_n.
		\]
		Due to \(\frac{k_i^{(n)}}{n}\to \lambda_i\), we can conclude \(x^{(n)}\to
		x\).
		
		\item
		Pick any \(x=(x^{(1)},\dots,x^{(\dims)})\in\conv(X_1)^\circ\). Since it is
		in the interior, there exists \(\epsilon>0\) such that
		\[
			x_{+\epsilon}
			:= (x^{(1)}+\epsilon, \dots, x^{(\dims)}+\epsilon) \in \conv(X_1)^\circ.
		\]
		By \ref{set-conv: seq} we know there exists a sequence
		\(x_n\in\bar{X}_n\) with \(x_n\to x_{+\epsilon}\), and therefore there exists
		\(n_0\in\nat\) such that for all \(n\ge n_0\), \(x_n\) is included in the
		\(\epsilon\) ball induced by the sup-norm around \(x_{+\epsilon}\). But this
		implies \(x^{(i)} \le x^{(i)}_n\) for all \(n\ge n_0\). As the \ref{eq: lower
		layer} property easily translates\footnote{
			the lower layer property essentially allows us to throw away surplus. It
			is intuitive that we can do that with the total output as well. To show
			this mathematically we only need to distribute this ``throwing away''
			action among the producers of the product we want to throw away.
		} to the
		Minkowski sum and the average production capabilities \(\bar{X}_n\), we
		conclude \(x\in\bar{X}_n\) for all \(n\ge n_0\).
		\qedhere
	\end{enumerate}
\end{enumerate}
\end{proof}

In general, the production sets \(X_i\) are obviously not equal. But assuming
they are drawn from some sort of probability distribution, it seems plausible
that \(\bar{X}_n\) would still converge to a convex set. But defining a
probability distribution over the sets with the \ref{eq: lower layer} property,
is not trivial to do. Later we will consider the special case of an
extended pure variable cost case (with setup costs), where the costs are random
variables.\fxnote{add reference}