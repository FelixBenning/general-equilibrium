\section{Finding the Right Prices}


The question remains: What is the right price? Should we average the
\(l^{(j)}_i\) somehow?

It turns out, that ensuring supply to be equal to demand
\(S=D\) leaves us with very little choice when it comes to prices anyway.
Let us see what happens for a fixed price vector. We want \(S=D\). This
obviously implies
\begin{equation}
	\label{eq: supply gdp = demand gdp}
	0 =\langle S-D,p\rangle =  \sum_{i=1}^n \langle S_i - D_i, p\rangle
\end{equation}
A market based system on the other hand ensures
\[
	0 = \langle S_i - D_i, p\rangle
	= \underbrace{\langle S_i, p\rangle}_{\text{income}}
	- \underbrace{\langle D_i,p\rangle}_{\text{expenses}}
\]
which is more than sufficient for the sums in Equation~\eqref{eq: supply gdp =
demand gdp} to be equal, but not sufficient for \(S=D\). In fact, it only
ensures that the price is orthogonal to the demand/supply mismatch \(S-D\).
If there are \(d\) products, that leaves a \(d-1\) hyperplane.

But as we will see, there are some price vectors, which do ensure \(S=D\).
We call these economic equilibria.

\begin{figure}
	\centering
	\includegraphics[width=0.7\textwidth]{images/consumption_increase_by_trade.jpeg}
	\caption{
		Product vectors on the \(d-1\) dimensional hyperplanes orthogonal to \(p\)
		cost the same amount and can therefore be exchanged (here \(d=2\), so the
		hyperplanes are lines). Trading at prices \(p\) enlarges the set of
		product vectors available for consumption for person \(i\).
	}
	\label{fig: consumption increase by trade}
\end{figure}
For a fixed price vector \(p\), let us consider the production and consumption
decision of person \(i\). As their expenses have to be smaller than their
income, it makes sense to maximize income for a given amount of labour \(L_i\).
Recall that our production options are \(X_i=X_i(L_i)\). So the maximal income
given \(L_i\) is
\[
	\tag{income}\label{eq: income}
	\mu(p, X_i) := \sup\{\langle p, x\rangle : x\in X_i\}.
\]
The function \(\mu(\cdot, X_i)\) in \(p\), is called the ``support function'' of
\(X_i\). It has some useful properties we will be grateful for later on. Our
utility optimization problem therefore becomes
\[
	\max_{L_i, y} u(1-L_i, y) \quad\text{subject to}\quad \langle y, p\rangle \le \mu(p, X_i(L_i))
\]
In Figure~\ref{fig: consumption increase by trade} we can see, how this
constraint is always weaker than \(y\in X_i(L_i)\), which is the constraint of
self-sufficiency. I.e. we get the following lemma.

\begin{lemma}[Trade is never harmful]
\[
	\underbrace{X_i(L_i)}_{\text{own production}}
	\subseteq \quad
	\underbrace{
		\{y\in\real_{\ge 0}^\dims: \langle y, p\rangle \le \mu(p, X_i)\}
	}_{\text{consumption options with trade}}
\]
\end{lemma}
\begin{proof}
	Choose an arbitrary \(y\in X_i\), then by definition of \(\mu\)
	\[
		\langle p, y\rangle \overset{y\in X_i}\le \sup\{\langle p, x\rangle : x\in
		X_i\} \overset{\text{def.}}= \mu(p, X_i),
	\]
	\(y\) is also in the set on the right.
\end{proof}




\begin{example}[Cloned Hermit Economy]
	If we clone our hermit in the previous example \(n\) times, the clones could
	start trading. Although they have very little reason to do so for now.
	Nevertheless let us first consider what equilibrium prices are possible
	before we modify \(X_i\) to give them a reason to do so.
\end{example}

\begin{figure}
	\includegraphics[width=0.58\textwidth]{images/hermit-decision-setup-cost.jpeg}
	\caption{
	fixed set-up time \(f^{(2)}\) for the production of good \(2\). This allows
	for more production of good \(1\) if \(x^{(2)}=0\). The dotted line
	represents the production frontier in the ``cloned hermit economy''.}
\end{figure}