\section{Support Function}

\begin{lemma}[Convexity]
	The support function \(\mu\) is convex.	
\end{lemma}
\begin{proof}
	Let \(p,q \in \real^\dims\) and \(\lambda\in[0,1]\), then
	\begin{align*}
		\mu(\lambda p + (1-\lambda)q , X)
		&= \sup\{\lambda \langle  p , x\rangle + (1-\lambda)\langle q, x\rangle : x\in X\}\\
		&\le \lambda \sup\{\langle p, x\rangle :x\in X\}
		+ (1-\lambda) \sup\{\langle q, y\rangle : y \in X\}\\
		&= \lambda \mu(p, X)+ (1-\lambda)\mu(q, X)
		\qedhere
	\end{align*}
\end{proof}

\begin{definition}[Subgradient]
	Let \(f:\real^\dims\to \real\) be a convex function. The subgradient
	\(\partial f\) of \(f\) is then defined as a set-valued function, by
	\[
		\partial f(x_0)
		:= \{
			v\in \real^\dims :
			f(x)-f(x_0) \ge \langle v, x-x_0\rangle
			\quad \forall x\in\real^\dims
		\}
	\]
\end{definition}

You can interpret the set of subgradients \(v\in\partial f(x_0)\) as the
set of slopes a lower tangent at \(x_0\) may have, by reordering the requirement
\[
	f(x) \ge \underbrace{f(x_0) + \langle v, x-x_0\rangle}_{\text{tangent}}.
\]

\begin{lemma}[Subgradients of Support Functions]
	\[
		\argmax_{x\in X}\langle p, X\rangle \subseteq \partial_p \mu(p, X)
	\]
\end{lemma}
\begin{proof}
	Let \(x^* \in \argmax_{x\in X}\langle p, X\rangle\). Then by definition
	\begin{equation}
		\label{eq: x^* is opt w.r.t. p}
		\mu(p, X) = \langle p, x^*\rangle
	\end{equation}
	and
	\begin{equation}
		\label{eq: support function is sup}
		\mu(q, X) \ge \langle q, x^*\rangle \quad \forall q\in\real^\dims.
	\end{equation}
	subtracting \eqref{eq: x^* is opt w.r.t. p} from \eqref{eq: support function is sup}
	results in
	\[
		\mu(q, X) - \mu(p, X)
		\ge \langle q, x^*\rangle - \langle p, x^*\rangle
		= \langle x^*, q-p\rangle.
	\]
	So by definition of the subgradient \(x^*\in \partial_p \mu(p, X)\).
\end{proof}
